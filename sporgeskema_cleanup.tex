% Options for packages loaded elsewhere
\PassOptionsToPackage{unicode}{hyperref}
\PassOptionsToPackage{hyphens}{url}
%
\documentclass[
]{article}
\usepackage{amsmath,amssymb}
\usepackage{lmodern}
\usepackage{ifxetex,ifluatex}
\ifnum 0\ifxetex 1\fi\ifluatex 1\fi=0 % if pdftex
  \usepackage[T1]{fontenc}
  \usepackage[utf8]{inputenc}
  \usepackage{textcomp} % provide euro and other symbols
\else % if luatex or xetex
  \usepackage{unicode-math}
  \defaultfontfeatures{Scale=MatchLowercase}
  \defaultfontfeatures[\rmfamily]{Ligatures=TeX,Scale=1}
\fi
% Use upquote if available, for straight quotes in verbatim environments
\IfFileExists{upquote.sty}{\usepackage{upquote}}{}
\IfFileExists{microtype.sty}{% use microtype if available
  \usepackage[]{microtype}
  \UseMicrotypeSet[protrusion]{basicmath} % disable protrusion for tt fonts
}{}
\makeatletter
\@ifundefined{KOMAClassName}{% if non-KOMA class
  \IfFileExists{parskip.sty}{%
    \usepackage{parskip}
  }{% else
    \setlength{\parindent}{0pt}
    \setlength{\parskip}{6pt plus 2pt minus 1pt}}
}{% if KOMA class
  \KOMAoptions{parskip=half}}
\makeatother
\usepackage{xcolor}
\IfFileExists{xurl.sty}{\usepackage{xurl}}{} % add URL line breaks if available
\IfFileExists{bookmark.sty}{\usepackage{bookmark}}{\usepackage{hyperref}}
\hypersetup{
  pdftitle={sporgeskema},
  pdfauthor={Julie Svinth Nielsen},
  hidelinks,
  pdfcreator={LaTeX via pandoc}}
\urlstyle{same} % disable monospaced font for URLs
\usepackage[margin=1in]{geometry}
\usepackage{color}
\usepackage{fancyvrb}
\newcommand{\VerbBar}{|}
\newcommand{\VERB}{\Verb[commandchars=\\\{\}]}
\DefineVerbatimEnvironment{Highlighting}{Verbatim}{commandchars=\\\{\}}
% Add ',fontsize=\small' for more characters per line
\usepackage{framed}
\definecolor{shadecolor}{RGB}{248,248,248}
\newenvironment{Shaded}{\begin{snugshade}}{\end{snugshade}}
\newcommand{\AlertTok}[1]{\textcolor[rgb]{0.94,0.16,0.16}{#1}}
\newcommand{\AnnotationTok}[1]{\textcolor[rgb]{0.56,0.35,0.01}{\textbf{\textit{#1}}}}
\newcommand{\AttributeTok}[1]{\textcolor[rgb]{0.77,0.63,0.00}{#1}}
\newcommand{\BaseNTok}[1]{\textcolor[rgb]{0.00,0.00,0.81}{#1}}
\newcommand{\BuiltInTok}[1]{#1}
\newcommand{\CharTok}[1]{\textcolor[rgb]{0.31,0.60,0.02}{#1}}
\newcommand{\CommentTok}[1]{\textcolor[rgb]{0.56,0.35,0.01}{\textit{#1}}}
\newcommand{\CommentVarTok}[1]{\textcolor[rgb]{0.56,0.35,0.01}{\textbf{\textit{#1}}}}
\newcommand{\ConstantTok}[1]{\textcolor[rgb]{0.00,0.00,0.00}{#1}}
\newcommand{\ControlFlowTok}[1]{\textcolor[rgb]{0.13,0.29,0.53}{\textbf{#1}}}
\newcommand{\DataTypeTok}[1]{\textcolor[rgb]{0.13,0.29,0.53}{#1}}
\newcommand{\DecValTok}[1]{\textcolor[rgb]{0.00,0.00,0.81}{#1}}
\newcommand{\DocumentationTok}[1]{\textcolor[rgb]{0.56,0.35,0.01}{\textbf{\textit{#1}}}}
\newcommand{\ErrorTok}[1]{\textcolor[rgb]{0.64,0.00,0.00}{\textbf{#1}}}
\newcommand{\ExtensionTok}[1]{#1}
\newcommand{\FloatTok}[1]{\textcolor[rgb]{0.00,0.00,0.81}{#1}}
\newcommand{\FunctionTok}[1]{\textcolor[rgb]{0.00,0.00,0.00}{#1}}
\newcommand{\ImportTok}[1]{#1}
\newcommand{\InformationTok}[1]{\textcolor[rgb]{0.56,0.35,0.01}{\textbf{\textit{#1}}}}
\newcommand{\KeywordTok}[1]{\textcolor[rgb]{0.13,0.29,0.53}{\textbf{#1}}}
\newcommand{\NormalTok}[1]{#1}
\newcommand{\OperatorTok}[1]{\textcolor[rgb]{0.81,0.36,0.00}{\textbf{#1}}}
\newcommand{\OtherTok}[1]{\textcolor[rgb]{0.56,0.35,0.01}{#1}}
\newcommand{\PreprocessorTok}[1]{\textcolor[rgb]{0.56,0.35,0.01}{\textit{#1}}}
\newcommand{\RegionMarkerTok}[1]{#1}
\newcommand{\SpecialCharTok}[1]{\textcolor[rgb]{0.00,0.00,0.00}{#1}}
\newcommand{\SpecialStringTok}[1]{\textcolor[rgb]{0.31,0.60,0.02}{#1}}
\newcommand{\StringTok}[1]{\textcolor[rgb]{0.31,0.60,0.02}{#1}}
\newcommand{\VariableTok}[1]{\textcolor[rgb]{0.00,0.00,0.00}{#1}}
\newcommand{\VerbatimStringTok}[1]{\textcolor[rgb]{0.31,0.60,0.02}{#1}}
\newcommand{\WarningTok}[1]{\textcolor[rgb]{0.56,0.35,0.01}{\textbf{\textit{#1}}}}
\usepackage{graphicx}
\makeatletter
\def\maxwidth{\ifdim\Gin@nat@width>\linewidth\linewidth\else\Gin@nat@width\fi}
\def\maxheight{\ifdim\Gin@nat@height>\textheight\textheight\else\Gin@nat@height\fi}
\makeatother
% Scale images if necessary, so that they will not overflow the page
% margins by default, and it is still possible to overwrite the defaults
% using explicit options in \includegraphics[width, height, ...]{}
\setkeys{Gin}{width=\maxwidth,height=\maxheight,keepaspectratio}
% Set default figure placement to htbp
\makeatletter
\def\fps@figure{htbp}
\makeatother
\setlength{\emergencystretch}{3em} % prevent overfull lines
\providecommand{\tightlist}{%
  \setlength{\itemsep}{0pt}\setlength{\parskip}{0pt}}
\setcounter{secnumdepth}{-\maxdimen} % remove section numbering
\ifluatex
  \usepackage{selnolig}  % disable illegal ligatures
\fi

\title{sporgeskema}
\author{Julie Svinth Nielsen}
\date{19/9/2021}

\begin{document}
\maketitle

\hypertarget{pre-processing}{%
\subsubsection{Pre-processing}\label{pre-processing}}

Her hentes dataen ind og bliver renset.

\begin{Shaded}
\begin{Highlighting}[]
\FunctionTok{library}\NormalTok{(tidyverse)}
\end{Highlighting}
\end{Shaded}

\begin{verbatim}
## -- Attaching packages --------------------------------------- tidyverse 1.3.1 --
\end{verbatim}

\begin{verbatim}
## v ggplot2 3.3.3     v purrr   0.3.4
## v tibble  3.1.1     v dplyr   1.0.5
## v tidyr   1.1.3     v stringr 1.4.0
## v readr   1.4.0     v forcats 0.5.1
\end{verbatim}

\begin{verbatim}
## -- Conflicts ------------------------------------------ tidyverse_conflicts() --
## x dplyr::filter() masks stats::filter()
## x dplyr::lag()    masks stats::lag()
\end{verbatim}

\begin{Shaded}
\begin{Highlighting}[]
\FunctionTok{library}\NormalTok{(readr)}
\FunctionTok{library}\NormalTok{(dplyr)}
\FunctionTok{library}\NormalTok{(plyr)}
\end{Highlighting}
\end{Shaded}

\begin{verbatim}
## ------------------------------------------------------------------------------
\end{verbatim}

\begin{verbatim}
## You have loaded plyr after dplyr - this is likely to cause problems.
## If you need functions from both plyr and dplyr, please load plyr first, then dplyr:
## library(plyr); library(dplyr)
\end{verbatim}

\begin{verbatim}
## ------------------------------------------------------------------------------
\end{verbatim}

\begin{verbatim}
## 
## Attaching package: 'plyr'
\end{verbatim}

\begin{verbatim}
## The following objects are masked from 'package:dplyr':
## 
##     arrange, count, desc, failwith, id, mutate, rename, summarise,
##     summarize
\end{verbatim}

\begin{verbatim}
## The following object is masked from 'package:purrr':
## 
##     compact
\end{verbatim}

Load in data

\begin{Shaded}
\begin{Highlighting}[]
\NormalTok{spg }\OtherTok{\textless{}{-}} \FunctionTok{read\_delim}\NormalTok{(}\StringTok{"complete.csv"}\NormalTok{, }\StringTok{";"}\NormalTok{, }
    \AttributeTok{escape\_double =} \ConstantTok{FALSE}\NormalTok{, }\AttributeTok{trim\_ws =} \ConstantTok{TRUE}\NormalTok{)}
\end{Highlighting}
\end{Shaded}

\begin{verbatim}
## 
## -- Column specification --------------------------------------------------------
## cols(
##   .default = col_character(),
##   `E-mail` = col_logical(),
##   `Samlet status - Ny` = col_double(),
##   `Samlet status - Distribueret` = col_double(),
##   `Samlet status - Nogen svar` = col_double(),
##   `Samlet status - Gennemf<U+663C><U+3E38>rt` = col_double(),
##   `Samlet status - Frafaldet` = col_double()
## )
## i Use `spec()` for the full column specifications.
\end{verbatim}

Create ID

\begin{Shaded}
\begin{Highlighting}[]
\NormalTok{spg}\SpecialCharTok{$}\NormalTok{ID }\OtherTok{\textless{}{-}} \FunctionTok{seq.int}\NormalTok{(}\FunctionTok{nrow}\NormalTok{(spg))}

\CommentTok{\# Make all answers numeric perhaps? Give them the number from 1 to 5}
\NormalTok{spg[spg }\SpecialCharTok{==} \StringTok{"Enig"}\NormalTok{] }\OtherTok{\textless{}{-}} \StringTok{"5"}
\NormalTok{spg[spg }\SpecialCharTok{==} \StringTok{"Delvis enig"}\NormalTok{] }\OtherTok{\textless{}{-}} \StringTok{"4"}
\NormalTok{spg[spg }\SpecialCharTok{==} \StringTok{"Hverken eller"}\NormalTok{] }\OtherTok{\textless{}{-}} \StringTok{"3"}
\NormalTok{spg[spg }\SpecialCharTok{==} \StringTok{"Delvis uenig"}\NormalTok{] }\OtherTok{\textless{}{-}} \StringTok{"2"}
\NormalTok{spg[spg }\SpecialCharTok{==} \StringTok{"Uenig"}\NormalTok{] }\OtherTok{\textless{}{-}} \StringTok{"1"}

\NormalTok{df }\OtherTok{\textless{}{-}}\NormalTok{ spg}



\NormalTok{df\_2 }\OtherTok{\textless{}{-}} \FunctionTok{gather}\NormalTok{(df, factor, response, intro\_1}\SpecialCharTok{:}\NormalTok{nudge\_4, }\AttributeTok{factor\_key =} \ConstantTok{TRUE}\NormalTok{)}



\NormalTok{df\_2 }\OtherTok{\textless{}{-}}\NormalTok{ df\_2 }\SpecialCharTok{\%\textgreater{}\%} 
  \FunctionTok{mutate}\NormalTok{(}\AttributeTok{category =} \FunctionTok{substr}\NormalTok{(factor, }\DecValTok{1}\NormalTok{,}\DecValTok{4}\NormalTok{)) }\SpecialCharTok{\%\textgreater{}\%} 
  \FunctionTok{mutate}\NormalTok{(}\AttributeTok{category =} \FunctionTok{as.factor}\NormalTok{(category))}

\NormalTok{df\_2}\SpecialCharTok{$}\NormalTok{category }\OtherTok{\textless{}{-}} \FunctionTok{revalue}\NormalTok{(df\_2}\SpecialCharTok{$}\NormalTok{category, }\FunctionTok{c}\NormalTok{(}\StringTok{"intr"} \OtherTok{=} \StringTok{"introduction"}\NormalTok{, }\StringTok{"oply"} \OtherTok{=} \StringTok{"oplysning"}\NormalTok{, }\StringTok{"insp"} \OtherTok{=} \StringTok{"inspiration"}\NormalTok{, }\StringTok{"nudg"}\OtherTok{=}\StringTok{"nudging"}\NormalTok{))}
\end{Highlighting}
\end{Shaded}

\hypertarget{illustrationer-af-de-forskelligie-kategorier}{%
\subsubsection{Illustrationer af de forskelligie
kategorier}\label{illustrationer-af-de-forskelligie-kategorier}}

\begin{Shaded}
\begin{Highlighting}[]
\NormalTok{df\_2 }\SpecialCharTok{\%\textgreater{}\%} 
  \FunctionTok{filter}\NormalTok{(category }\SpecialCharTok{\%in\%} \FunctionTok{c}\NormalTok{(}\StringTok{"oplysning"}\NormalTok{, }\StringTok{"inspiration"}\NormalTok{, }\StringTok{"nudging"}\NormalTok{)) }\SpecialCharTok{\%\textgreater{}\%} 
  \FunctionTok{ggplot}\NormalTok{(}\FunctionTok{aes}\NormalTok{(category, }\FunctionTok{as.numeric}\NormalTok{(response), }\AttributeTok{fill =}\NormalTok{ category))}\SpecialCharTok{+}\FunctionTok{geom\_bar}\NormalTok{( }\AttributeTok{stat =} \StringTok{"summary"}\NormalTok{)}\SpecialCharTok{+} 
  \FunctionTok{xlab}\NormalTok{(}\StringTok{"De tre kategorier"}\NormalTok{)}\SpecialCharTok{+} \FunctionTok{ylab}\NormalTok{(}\StringTok{"Gennemsnittet af besvarelserne"}\NormalTok{)}
\end{Highlighting}
\end{Shaded}

\begin{verbatim}
## No summary function supplied, defaulting to `mean_se()`
\end{verbatim}

\includegraphics{sporgeskema_cleanup_files/figure-latex/unnamed-chunk-4-1.pdf}

\begin{Shaded}
\begin{Highlighting}[]
\NormalTok{df\_2 }\SpecialCharTok{\%\textgreater{}\%} 
  \FunctionTok{filter}\NormalTok{(category }\SpecialCharTok{\%in\%} \FunctionTok{c}\NormalTok{(}\StringTok{"oplysning"}\NormalTok{, }\StringTok{"inspiration"}\NormalTok{, }\StringTok{"nudging"}\NormalTok{)) }\SpecialCharTok{\%\textgreater{}\%} 
  \FunctionTok{mutate}\NormalTok{(}\AttributeTok{response =} \FunctionTok{as.numeric}\NormalTok{(response)) }\SpecialCharTok{\%\textgreater{}\%} 
  \FunctionTok{group\_by}\NormalTok{(category) }\SpecialCharTok{\%\textgreater{}\%} 
\NormalTok{  dplyr}\SpecialCharTok{::}\FunctionTok{summarise}\NormalTok{(}\AttributeTok{mean =} \FunctionTok{mean}\NormalTok{(response, }\AttributeTok{na.rm =} \ConstantTok{TRUE}\NormalTok{))}
\end{Highlighting}
\end{Shaded}

\begin{verbatim}
## # A tibble: 3 x 2
##   category     mean
##   <fct>       <dbl>
## 1 inspiration  3.60
## 2 nudging      3.40
## 3 oplysning    3.59
\end{verbatim}

\hypertarget{overordnede-gennemsnit-ved-de-forskellige-udsagn.}{%
\subsection{Overordnede gennemsnit ved de forskellige
udsagn.}\label{overordnede-gennemsnit-ved-de-forskellige-udsagn.}}

Dette overblik er undersøgt dybere længere nede.

\begin{Shaded}
\begin{Highlighting}[]
\NormalTok{df\_2 }\SpecialCharTok{\%\textgreater{}\%} 
  \FunctionTok{filter}\NormalTok{(category }\SpecialCharTok{\%in\%} \FunctionTok{c}\NormalTok{(}\StringTok{"oplysning"}\NormalTok{, }\StringTok{"inspiration"}\NormalTok{, }\StringTok{"nudging"}\NormalTok{)) }\SpecialCharTok{\%\textgreater{}\%} 
  \FunctionTok{ggplot}\NormalTok{(}\FunctionTok{aes}\NormalTok{(factor, }\FunctionTok{as.numeric}\NormalTok{(response), }\AttributeTok{fill =}\NormalTok{ category))}\SpecialCharTok{+}\FunctionTok{geom\_bar}\NormalTok{(}\AttributeTok{stat =} \StringTok{"summary"}\NormalTok{)}
\end{Highlighting}
\end{Shaded}

\begin{verbatim}
## No summary function supplied, defaulting to `mean_se()`
\end{verbatim}

\includegraphics{sporgeskema_cleanup_files/figure-latex/unnamed-chunk-5-1.pdf}

\hypertarget{intoduktion}{%
\subsubsection{Intoduktion}\label{intoduktion}}

\begin{Shaded}
\begin{Highlighting}[]
\CommentTok{\# Illustrate the different groups and the answers within this area {-} intro, oplysning, inspiration, nudging.}

\NormalTok{df\_2 }\SpecialCharTok{\%\textgreater{}\%} 
  \FunctionTok{filter}\NormalTok{(factor }\SpecialCharTok{==} \StringTok{"intro\_1"}\NormalTok{) }\SpecialCharTok{\%\textgreater{}\%} 
  \FunctionTok{ggplot}\NormalTok{(}\FunctionTok{aes}\NormalTok{(}\FunctionTok{as.factor}\NormalTok{(response), }\AttributeTok{fill =}\NormalTok{ response))}\SpecialCharTok{+}\FunctionTok{geom\_bar}\NormalTok{()}
\end{Highlighting}
\end{Shaded}

\includegraphics{sporgeskema_cleanup_files/figure-latex/unnamed-chunk-6-1.pdf}

\begin{Shaded}
\begin{Highlighting}[]
\NormalTok{df\_2 }\SpecialCharTok{\%\textgreater{}\%} 
  \FunctionTok{filter}\NormalTok{(factor }\SpecialCharTok{\%in\%} \FunctionTok{c}\NormalTok{(}\StringTok{"intro\_2"}\NormalTok{, }\StringTok{"intro\_3"}\NormalTok{, }\StringTok{"intro\_4"}\NormalTok{, }\StringTok{"intro\_5"}\NormalTok{)) }\SpecialCharTok{\%\textgreater{}\%} 
  \FunctionTok{ggplot}\NormalTok{(}\FunctionTok{aes}\NormalTok{(factor, }\FunctionTok{as.numeric}\NormalTok{(response), }\AttributeTok{fill =}\NormalTok{ factor))}\SpecialCharTok{+}\FunctionTok{geom\_bar}\NormalTok{( }\AttributeTok{stat =} \StringTok{"summary"}\NormalTok{)}\SpecialCharTok{+} 
  \FunctionTok{xlab}\NormalTok{(}\StringTok{"Introduktionsudsagn"}\NormalTok{)}\SpecialCharTok{+} \FunctionTok{ylab}\NormalTok{(}\StringTok{"Gennemsnittet af besvarelserne"}\NormalTok{)}
\end{Highlighting}
\end{Shaded}

\begin{verbatim}
## No summary function supplied, defaulting to `mean_se()`
\end{verbatim}

\includegraphics{sporgeskema_cleanup_files/figure-latex/unnamed-chunk-6-2.pdf}

\begin{Shaded}
\begin{Highlighting}[]
\FunctionTok{count}\NormalTok{(spg}\SpecialCharTok{$}\NormalTok{intro\_1)}
\end{Highlighting}
\end{Shaded}

\begin{verbatim}
##      x freq
## 1   Ja  352
## 2  Nej   34
## 3 <NA>    3
\end{verbatim}

\hypertarget{introduktionsudsagn}{%
\subsection{Introduktionsudsagn}\label{introduktionsudsagn}}

intro\_2: Jeg går op i at handle klimavenligt

intro\_3:Jeg prioriterer lavere priser over det klimabevidste valg

intro\_4:Jeg er villig til at ændre mine indkøbsvaner, så de bliver mere
klimavenlige

intro\_5:Der er så mange hensyn at tage i forhold til indkøb, så jeg
prioriterer ikke klimaet

\hypertarget{hvad-betyder-resultaterne}{%
\subsection{Hvad betyder
resultaterne?}\label{hvad-betyder-resultaterne}}

\begin{enumerate}
\def\labelenumi{\arabic{enumi}.}
\setcounter{enumi}{1}
\item
  Her indikerer gennemsnittet, at en forholdsvis høj andel af kunderne
  går op i at handle klimavenligt.
\item
  Her ligger gennemsnittet omkring hverken eller, hvilket indikerer, at
  prisen ikke dominerer særlig meget over det klimabevidste valg.
\item
  Her indikerer gennemsnittet, at en høj andel af kunderne ville være
  klar på at ændre sine vaner og gøre indkøbene mere klimabevidste.
\item
  Her indikerer gennemsnittet, at de fleste kunder ikke oplever, at der
  er forhold, som afgrænser muligheden for at handle klimabevidst.
\end{enumerate}

\hypertarget{konklusion}{%
\subsection{Konklusion}\label{konklusion}}

I de indledende spørgsmål ser vi, at kunderne oplever, at de selv går op
i at handle klimavenligt. Dette vidner om, at en større andel af
kunderne har lyst til at opnå et højere klimevenligt niveau. Dette
understøttes også af gennemsnittet på intro\_4, hvor en stor del af
kunderne gerne vil ændre deres indkøbsvaner, så de bliver mere
klimavenlige. Intensionerne er det altså, hvilket er positivt.

\hypertarget{oplysning}{%
\subsubsection{Oplysning}\label{oplysning}}

\begin{Shaded}
\begin{Highlighting}[]
\NormalTok{df\_2 }\SpecialCharTok{\%\textgreater{}\%} 
  \FunctionTok{filter}\NormalTok{(category }\SpecialCharTok{\%in\%} \FunctionTok{c}\NormalTok{(}\StringTok{"oplysning"}\NormalTok{)) }\SpecialCharTok{\%\textgreater{}\%} 
  \FunctionTok{ggplot}\NormalTok{(}\FunctionTok{aes}\NormalTok{(factor, }\FunctionTok{as.numeric}\NormalTok{(response), }\AttributeTok{fill =}\NormalTok{ factor))}\SpecialCharTok{+}\FunctionTok{geom\_bar}\NormalTok{(}\AttributeTok{stat =} \StringTok{"summary"}\NormalTok{)}\SpecialCharTok{+} 
  \FunctionTok{xlab}\NormalTok{(}\StringTok{"Udsagn omkring oplysning"}\NormalTok{)}\SpecialCharTok{+} \FunctionTok{ylab}\NormalTok{(}\StringTok{"Gennemsnittet af besvarelserne"}\NormalTok{)}
\end{Highlighting}
\end{Shaded}

\begin{verbatim}
## No summary function supplied, defaulting to `mean_se()`
\end{verbatim}

\includegraphics{sporgeskema_cleanup_files/figure-latex/unnamed-chunk-7-1.pdf}
\#\# Udsagnene omkring oplysning 1. Jeg føler mig godt oplyst om hvilke
varer der er klimavenlige

\begin{enumerate}
\def\labelenumi{\arabic{enumi}.}
\setcounter{enumi}{1}
\item
  Jeg opsøger tit information om klimaaftrykket på varernes mærkater
\item
  Jeg savner mærkater eller skilte, der fremhæver klimavenlige varer i
  supermarkedet
\item
  Jeg prioriterer at spise sæsonbestemte madvarer (fx frugt og grønt)
\item
  Jeg prioriterer at spise danske varer
\end{enumerate}

\hypertarget{hvad-betyder-resultaterne-1}{%
\subsection{Hvad betyder
resultaterne?}\label{hvad-betyder-resultaterne-1}}

\begin{enumerate}
\def\labelenumi{\arabic{enumi}.}
\tightlist
\item
  Lige omkring midten. Dette betyder at folk gennemsnitligt er hverken
  uenig eller enig med udsagnet om, at de føler sig godt oplyst om
  hvilke varer, der er klimavenlige.
\item
  Her ligger gennemsnittet imellem delvis uenig og hverken eller,
  hvilket betyder, at kunder ikke bruger så meget energi på at opsøge
  information om klimaaftrykket på varemærkater.
\item
  Her ser vi et gennemsnit, der ligger lidt over delvis enig. Dette
  indikerer, at kunder i høj grad mangler mærkater eller skilte, der
  fremhæver varer i supermarkedet.
\item
  Her ligger gennemsnittet lige over delvis enig, hvilket indikerer, at
  kunder i høj grad prioriterer at spise sæsonbestemt.
\item
  Her ligger gennemsnittet lige over delvis enig, hvilket indikerer at
  kunder prioriterer danske varer.
\end{enumerate}

\hypertarget{konklusion-1}{%
\subsection{Konklusion}\label{konklusion-1}}

Det der sprigner i øjnene her er, at kunderne efterspørger mærkater
eller skiltning i supermarkederne, så de kan blive mere oplyste om det
klimavenlige valg. Derudover ses der en tendens ved at handle
forholdsvis sæsonbestemt og lokalt.

\hypertarget{inspiration}{%
\subsubsection{Inspiration}\label{inspiration}}

\begin{Shaded}
\begin{Highlighting}[]
\NormalTok{df\_2 }\SpecialCharTok{\%\textgreater{}\%} 
  \FunctionTok{filter}\NormalTok{(category }\SpecialCharTok{\%in\%} \FunctionTok{c}\NormalTok{(}\StringTok{"inspiration"}\NormalTok{)) }\SpecialCharTok{\%\textgreater{}\%} 
  \FunctionTok{ggplot}\NormalTok{(}\FunctionTok{aes}\NormalTok{(factor, }\FunctionTok{as.numeric}\NormalTok{(response), }\AttributeTok{fill =}\NormalTok{ factor))}\SpecialCharTok{+}\FunctionTok{geom\_bar}\NormalTok{(}\AttributeTok{stat =} \StringTok{"summary"}\NormalTok{)}\SpecialCharTok{+} 
  \FunctionTok{xlab}\NormalTok{(}\StringTok{"Udsagn omkring inspiration"}\NormalTok{)}\SpecialCharTok{+} \FunctionTok{ylab}\NormalTok{(}\StringTok{"Gennemsnittet af besvarelserne"}\NormalTok{)}
\end{Highlighting}
\end{Shaded}

\begin{verbatim}
## No summary function supplied, defaulting to `mean_se()`
\end{verbatim}

\includegraphics{sporgeskema_cleanup_files/figure-latex/unnamed-chunk-8-1.pdf}
\#\# Udsagnene omkring inspiration 1. Jeg går op i at indrette min kost,
så den bliver mere klimavenlig

\begin{enumerate}
\def\labelenumi{\arabic{enumi}.}
\setcounter{enumi}{1}
\item
  Jeg synes ikke, at udvalget af klimavenlige produkter er stort nok
\item
  Jeg mangler inspiration til mere klimavenlige alternativer og
  opskrifter
\item
  Jeg synes, at det er svært at ændre mine madvaner, så de bliver mere
  klimavenlige
\end{enumerate}

\hypertarget{hvad-betyder-resultaterne-2}{%
\subsection{Hvad betyder
resultaterne?}\label{hvad-betyder-resultaterne-2}}

\begin{enumerate}
\def\labelenumi{\arabic{enumi}.}
\tightlist
\item
  Her ligger gennemsnittet mellem hverken eller og delvis enig. Dette
  indikerer at kunder er forholdsvis enige i at indrette deres kost mere
  klimavenligt.
\item
  Her ligger gennemsnittet mellem hverken eller og delvis enig, hvilket
  indikerer, at der er en lille overvægt af kunder, som ønsker et større
  udvalg af klimavenlige produkter.
\item
  Her ligger gennemsnittet lige omkring delvis enig, hvilket indikerer,
  at kunder i forholdsvis høj grad mangler mere inspiration til
  klimavenlige retter.
\item
  Her ligger gennemsnittet imellem hverken eller og delvis enig, hvilket
  giver en lille overvægt af kunder, der finder det svært at skabe mere
  klimavenlige madvaner.
\end{enumerate}

\hypertarget{konklusion-2}{%
\subsection{Konklusion}\label{konklusion-2}}

Kunderne ønsker gerne mere inspiration til klimavenlige retter. Der er
også en lille overvægt af kunder, som forsøger at indrette deres
måltider efter mere klimavenlige retningslinjer.

\hypertarget{nudging}{%
\subsubsection{Nudging}\label{nudging}}

\begin{Shaded}
\begin{Highlighting}[]
\NormalTok{df\_2 }\SpecialCharTok{\%\textgreater{}\%} 
  \FunctionTok{filter}\NormalTok{(category }\SpecialCharTok{\%in\%} \FunctionTok{c}\NormalTok{(}\StringTok{"nudging"}\NormalTok{)) }\SpecialCharTok{\%\textgreater{}\%} 
  \FunctionTok{ggplot}\NormalTok{(}\FunctionTok{aes}\NormalTok{(factor, }\FunctionTok{as.numeric}\NormalTok{(response), }\AttributeTok{fill =}\NormalTok{ factor))}\SpecialCharTok{+}\FunctionTok{geom\_bar}\NormalTok{(}\AttributeTok{stat =} \StringTok{"summary"}\NormalTok{)}\SpecialCharTok{+} 
  \FunctionTok{xlab}\NormalTok{(}\StringTok{"Udsagn omkring nudging"}\NormalTok{)}\SpecialCharTok{+} \FunctionTok{ylab}\NormalTok{(}\StringTok{"Gennemsnittet af besvarelserne"}\NormalTok{)}
\end{Highlighting}
\end{Shaded}

\begin{verbatim}
## No summary function supplied, defaulting to `mean_se()`
\end{verbatim}

\includegraphics{sporgeskema_cleanup_files/figure-latex/unnamed-chunk-9-1.pdf}

\hypertarget{udsagnene-omkring-nudging}{%
\subsection{Udsagnene omkring nudging}\label{udsagnene-omkring-nudging}}

\begin{enumerate}
\def\labelenumi{\arabic{enumi}.}
\item
  Jeg synes, at det er uoverskueligt at finde klimavenlige produkter i
  supermarkedet
\item
  Jeg prioriterer at indkøb skal gå hurtigt, hvilket sommetider gør
  valget af varer mere tilfældig
\item
  Jeg bruger ofte ekstra tid på at finde klimavenlige alternativer
\item
  Jeg synes, at klimavenlige varer skal placeres mere tilgængeligt end
  ikke-klimavenlige varer i supermarkedet
\end{enumerate}

\hypertarget{hvad-betyder-resultaterne-3}{%
\subsection{Hvad betyder
resultaterne?}\label{hvad-betyder-resultaterne-3}}

\begin{enumerate}
\def\labelenumi{\arabic{enumi}.}
\item
  Her indikerer gennemsnittet, at der er en lille overvægt af kunder,
  der synes, det er uoverskueligt at finde klimavenlgie produkter i
  supermarkedet.
\item
  Her ligger gennemsnittet lige omkring hverken eller, hvilket betyder,
  at hurtige indkøb ikke har så stor betydning for kundernes valg af
  varer.
\item
  Her ligger gennemsnitter imellem delvis uenig og hverken eller,
  hvilket indikerer at kunder formentlig ikke bruger ekstra tid på at
  finde klimavenlige alternativer.
\item
  Her ligger gennemsnittet lige omkrign delvig enig, hvilket indikerer
  at kunder er enige med ideen om, at klimavenlige produkter skal
  fremhæves mere end ikke-klimavenlige varer.
\end{enumerate}

\hypertarget{konklusion-3}{%
\subsection{Konklusion}\label{konklusion-3}}

Her ser vi i høj grad, at kunderne støtter op omkring, at klimavenlige
varer skal være mere tilgængelige sammenlignet med ikke-klimavenlige
varer. Resultaterne viser også, at kunderne ikke bruger ekstra tid på at
finde klimavenlige alternativer, så dette skal ikke forventes af dem.
Der er ligeledes en lille overvægt at kunder, der finder det svært at
navigere i, hvor man finder disse klimavenlige varer.

\end{document}
